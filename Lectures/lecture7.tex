\documentclass[main]{subfiles}

\begin{document}
%Set chapter counter as week-1
\chapPreamble{7}{March 10, 2023}{Isoperimetric Inequalities and Applications}
%Set chapter name

\lecture{Siva Athreya}{Atreya Choudhury}

The focus of this chapter is to look at how the geometry of weighted graph affects the properties of the corresponding random walk on it.

\begin{definition}[Isoperimetric Inequality]
	Let $A, B \subseteq V, \; \mu_E(A,B) = \sum_{x \in A} \sum_{y \in B} \mu_{xy}$
	and $\psi: \R_+ \to \R_+$ be an increasing function.

	$(\Gamma,\;\mu)$ is said to satisfy the $\psi-$isoperimetric inequality if $\exists\;c_0>0$ such that
	\[\frac{\mu_E(A,V\smallsetminus A)}{\psi(\mu(A))} \geq \frac{1}{c_0} \hspace{3em} \forall A \subseteq V \text{ and }\Mod{A} < \infty\]
	If a weighted graph satisfies the $\psi-$isoperimetric inequality, we say it has the $I_\psi$ property.

	A graph is said to have the property $I_\alpha$ for $\alpha \in [0, \infty)$ when $\psi(t) = t^{1-\frac{1}{\alpha}}$
	and said to have the property $I_\infty$ when $\psi(t) = t$
\end{definition}

\ex
$\R^d$. We look at $A = B(0,r)$
\begin{align*}
	S_B                                       \equiv \text{surface area of A} & = c_d r^{d-1}         \\
	V_B                                       \equiv \text{volume of A}       & = \widetilde{c_d} r^d \\
	\therefore \frac{S_B}{V_B^\frac{d-1}{d}}                                  & \geq \frac{1}{c_0}
\end{align*}

We can take $\psi(t) = t^{1-\frac{1}{d}}$

Show that $Z^d$ has the $I_d$ property for all such A such that $\Mod{A} < \infty$\\

\ex
$\prod_2$, the binary tree has the $I_\infty$ property with $c_0 = 3$
\begin{obs}
	If $(\Gamma,\;\mu)$ satisfies $I_{\alpha+\delta}$, then it satisfies $I_\alpha$
\end{obs}
\begin{definition}[Nash Inequality]
	$\alpha \in [1, \infty),\;(\Gamma,\;\mu)$ is said to have the property $N_\alpha$\\
	if $\forall f \in \L^1(V) \cap \L^2(V)$,
	\[\SE(f,f) \geq C_N\norm{f}{1}^{-\frac{4}{\alpha}}\norm{f}{2}^{2+\frac{4}{\alpha}}\]
\end{definition}
\newpage
\begin{remark}[]
	\begin{enumerate}
		\item $(\Gamma,\;\mu)$ satisfies $I_\alpha$ for $\alpha \in [1, \infty)$
		      $\implies (\Gamma,\;\mu)$ satisfies $N_\alpha$
		\item $Z^d$ satisfies $N_\alpha$
	\end{enumerate}
\end{remark}
\begin{theorem}
	Let $\alpha \geq 1.$ Then the following are equivalent
	\begin{enumerate}
		\item $(\Gamma,\;\mu)$ satisfies $N_\alpha$
		\item $\exists\;C_H > 0$ such that
		      \[p_n(x,x) \leq \frac{C_H}{(n\vee1)^\frac{\alpha}{2}} \hspace{3em} \forall\;n \geq 0\text{ and } x \in V\]
		\item $\exists\;C_H' > 0$ such that
		      \[p_n(x,y) \leq \frac{C_H'}{(n\vee1)^\frac{\alpha}{2}} \hspace{3em} \forall\;n \geq 0\text{ and } x,y \in V\]
	\end{enumerate}
\end{theorem}
\begin{corollary}
	\label{cor:ubounds}
	\begin{enumerate}
		\item Suppose $(\Gamma,\;\mu)$ satisfies $I_\alpha.$ Then, $\exists\;C > 0$ such that
		      \[p_n(x,y) \leq \frac{C}{(n\vee1)^\frac{\alpha}{2}} \hspace{3em} \forall\;n \geq 0\text{ and } x,y \in V\]
		\item Let V be infinite and $\mu_{xy} \geq c_0 > 0\;\;\forall\;x \sim y.$ Then, $\exists\;C_1 > 0$ such that
		      \[p_n(x,y) \leq \frac{C_1}{(n\vee1)^\frac{1}{2}} \hspace{3em} \forall\;n \geq 0\text{ and } x,y \in V\]
	\end{enumerate}
\end{corollary}
\begin{remark}
	\begin{enumerate}
		\item $p_n(x,x) \equiv$ on-diagonal bounds
		\item Theorem provides global upper bounds
		\item part b of corollary \ref{cor:ubounds} applied to $V = \Z$\\
		      $\implies$ the shortest possible on-diagonal upper bounds with natural weights
		\item Let $\Gamma = \Z^d$ have natural weights $\mu_{xy}^{(0)}$ and $\Gamma' = \Z^d$ have natural weights $\mu_{xy}^{(1)}$ such that $\mu_{xy}^{(1)} \geq c_0 \mu_{xy}^{(0)}$
		      Let $(\Gamma,\;\mu^0)$ satisfy $N_d$\\
		      $\implies (\Gamma',\;\mu^1)$ satisfies $N_d$\\
		      $\implies$ the upper bound of the theorem holds
		\item $\Gamma = \Z^d \cup_{(0, \ldots, 0)} \Z^d$\\
		      $\implies \Gamma$ also satisfies $N^d$
		\item \ref{cor:ubounds} does not give us any information on upper bounds when we fix $n \geq 0$ and let $d(x,y)$ get large.
	\end{enumerate}
\end{remark}
\begin{theorem}
	Let $(\Gamma,\;\mu)$ be a weighted graph. Then,
	\[p_n(x,y) \leq \frac{2}{\sqrt{\mu_x\mu_y}}e^{-\frac{d(x,y)^2}{2n}} \hspace{3em} \forall\;x,y \in V\text{ and }n \geq 1\]
\end{theorem}
\ex Consequences for $\Z^d$

We expect
\[p_n(x,y) \leq \frac{c_1}{n^\frac{d}{2}}e^{-c_2\frac{d(x,y)^2}{n}}\]

$\Z^d$ satisfies $I_d \implies \Z^d$ satisfies $N_d \stackrel{\ref{cor:ubounds}}{\implies} p_n(x,y) \leq \frac{c}{n^\frac{d}{2}} \hspace{3em} \forall\; x,y \in V$ and $n \geq 1$

\[\therefore\;p_n(x,y) \leq \frac{c}{n^\frac{d}{2}} \leq \frac{c}{n^\frac{d}{2}} e^{-\frac{d(x,y)^2}{n}} \hspace{3em}\text{ when }d(x,y)\leq\sqrt{n}\]

When, $d(x,y) \geq \sqrt{2dn\log n},$
\[p_n(x,y) \leq c_1 e^{-\frac{d(x,y)^2}{n}} = c_1 e^{-\frac{2c_2}{4}\frac{d(x,y)^2}{n}} e^{-\frac{2c_2}{4}\frac{d(x,y)^2}{n}} \leq \frac{\widetilde{c_1}}{n^\frac{d}{2}} e^{-\frac{c_2^2d(x,y)^2}{n}}\]
\begin{definition}
	$(\Gamma,\;\mu)$ is said to have \textbf{polynomial volume growth} if $\exists\;C_V$ and $\theta$ such that
	\[\max\{\Mod{B(x,r)},\;\mu(B(x,r))\} \leq C_Vr^\theta \hspace{3em} \forall\;x \in V\text{ and }r \geq 1\]
\end{definition}
\begin{lemma}
	$(\Gamma,\;\mu)$ has polynomial volume growth with index $\theta.$ Then,
	\[\P^x(d(x, X_n) > r) \leq cr^\theta e^{-\frac{r^2}{4n}}\]
	This implies $\exists\;c_2>0$ such that
	\[d(x, X_n) \leq c_2 \sqrt{n\log n} \hspace{3em} \forall\text{ large }n\text{ w.p. }1\]
	\begin{proof}
		We define $\SD_k = B(x,2^kr) \smallsetminus B(x,2^{k-1}r)$
		\begin{align*}
			\P^x(d(x, X_n) > r) & \stackrel{Ex}{=} \sum_{k=1}^\infty \sum_{y \in \SD_k} p_n(x,y)\mu_x                                                           \\
			                    & \stackrel{Ex}{\leq} \sum_{k=1}^\infty \sum_{y \in \SD_k} \frac{2}{\sqrt{\mu_x}}\sqrt{\mu_y}e^{-\frac{(2^{k-1}r)^2}{2n}}       \\
			                    & = \sum_{k=1}^\infty \frac{2}{\sqrt{\mu_x}} e^{-\frac{(2^{k-1}r)^2}{2n}} \sum_{y \in \SD_k} \sqrt{\mu_y}                       \\
			                    & \stackrel{Ex}{\leq} \sum_{k=1}^\infty \frac{2}{\sqrt{\mu_x}} e^{-\frac{(2^{k-1}r)^2}{2n}} \sqrt{\Mod{\SD_k}}\sqrt{\mu(\SD_k)} \\
			                    & \stackrel{Ex}{\leq} \sum_{k=1}^\infty \frac{2}{\sqrt{\mu_x}} e^{-\frac{(2^{k-1}r)^2}{2n}} c(2^kr)^\theta                      \\
		\end{align*}
	\end{proof}
\end{lemma}
\end{document}
