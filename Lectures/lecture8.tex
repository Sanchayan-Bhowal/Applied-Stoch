\documentclass[main]{subfiles}

\begin{document}
%Set chapter counter as week-8
\chapPreamble{8}{March 10, 2023}{Large Deviations for Random Walks}
%Set chapter name

\lecture{Siva Athreya}{Venkat Trivikram, Srivatsa B}

\subsection*{Nash Inequality (continued)}
\begin{theorem}
Let $(\Gamma, \mu)$ be a weighted graph, and let $\alpha\ge1$. TFAE.
\begin{itemize}
\item[(a)](Nash Inequality) $(\Gamma, \mu)$ satisfies $(N_{\alpha})$
\item[(b)](On Diagonal Bounds) There exists $C_H>0 $ such that for every $x\in V$ and $n\ge0$ \[p_n(x,x)\le\frac{C_H}{(n\lor 1)^{\alpha/2}}\]
\item[(c)](Off Diagonal Bounds) There exists $C_H'>0 $ such that for every $x,y\in V$ and $n\ge0$ \[p_n(x,y)\le\frac{C_H'}{(n\lor 1)^{\alpha/2}}\]
\end{itemize}
\end{theorem}
\begin{proof}
We provide only a sketch of the proof. From Worksheet 2, $(a) \implies (b)$  holds, and $(c) \implies (b)$ is trivial. First, we show $(b) \implies (c)$. So assume $(b)$.
Let $m\ge0$. If $n$ is even, with $n=2m$, then for any $x,y\in V$, we have
\[p_{2m}(x,x)\le\frac{C_H}{(2m\lor 1)^{\alpha/2}}\quad\text{and}\quad p_{2m}(y,y)\le\frac{C_H}{(2m\lor 1)^{\alpha/2}}\]
As an exercise, show that $p_{2m}(x,y)\le\sqrt{p_{2m}(x,x)p_{2m}(y,y)}$, and using this, we get $(b) \implies (c)$ with $C_H'=C_H$. If $n=2m+1$, then, since $p_{2m+1}(x,y)\le\sqrt{p_{2m}(x,x)p_{2m+2}(y,y)}$ (by a similar exercise), we get \[p_{2m+1}(x,y)\le\sqrt{\frac{C_H^2}{(2m\lor 1)^{\alpha/2}(2m+2\lor 1)^{\alpha/2}}}\le \frac{C'}{(2m+1\lor 1)^{\alpha/2}}\] for some $C'>0$. To show the last inequality above, use the fact that there exists $C_{\alpha}>0$ such that $(2m)^{\alpha/2}(2m+2)^{\alpha/2}\le C_{\alpha}(2m+1)^{\alpha/2}$ (details left as exercises). Thus $(b)\implies (c)$.\\
Now, we show $(c)\implies (a)$. Assuming $(c)$, observe that (by taking supremum over $x\in V$) \[\Mod{P_nf(x)} \le \sum_{y\in V}p_n(x,y)\Mod{f(y)}\mu_y\implies \norm{P_nf}{\infty} \le \frac{C_H}{(n\lor 1)^{\alpha/2}}\norm{f}{1}\]
\begin{equation} \label{eq:1}
	\text{and\quad}\norm{P_nf}{2}^2=\ip{P_nf}{P_nf}=\ip{P_{2n}f}{f}\le\norm{P_{2n}f}{\infty}\norm{f}{1}\le \frac{C_H}{(2n\lor 1)^{\alpha/2}}\norm{f}{1}^2
\end{equation}
Now, we make use of the following inequality - (verify!)
\[\SE(f, f)\ge \frac{1}{2n}[\norm{f}{2}^2-\norm{P_nf}{2}^2]\]
Using this, and (8.1), we get
\[\SE(f,f) \ge \frac{1}{2n}\bigg[\norm{f}{2}^2-\frac{C_H}{(2n\lor 1^{\alpha/2})}\norm{f}{1}^2\bigg]\]
WLOG, assume $\norm{f}{1}=1$, and choose smallest possible $k$ such that
\[\frac{C_H}{(2n\lor 1)^{\alpha/2}}\le\frac{\norm{f}{2}^2}{2}\text{\quad so that \quad}\SE(f,f)\ge\frac{1}{4k}\norm{f}{2}^2\]
Since $k\ge 1$, we have $k^{-\alpha/2}\le C^2\norm{f}{2}^2$ for some $C>0$, and hence $k^{-\alpha/2}\le C\norm{f}{2}$. Therefore,
\[\SE(f,f)\ge \frac{C_2\norm{f}{2}^2}{\norm{f}{2}^{\frac{4}{\alpha}}} = C_2\norm{f}{2}^{2-4/\alpha}\implies (N_{\alpha})\]
\end{proof}

\subsection*{Carne-Varopoulos Bound}
We begin with a few lemmas and some results involving Chebyshev polynomials.
\begin{lemma}
	Let $\{S_n\}_{n\ge 0}$ denote the simple symmetric random walk on $\Z$ with $S_0=0$. Then
	\begin{itemize}
		\item[(a)] \[\P(S_n\ge D)\le \exp\bigg(-\frac{D^2}{2n}\bigg)\]
		\item[(b)] \[\E[\lambda^{S_n}]=\sum_{r\in\Z}\lambda^r\P(S_n=r)=2^{-n}\sum_{r=0}^n\binom{n}{r}\bigg(\frac{1}{\lambda}\bigg)^{2n-r}\]
	\end{itemize}
\end{lemma}
\begin{proof}
	$(a)$ was given in Worksheet 2, and $(b)$ is trivial using results from Week $1$.
\end{proof}
\begin{definition}
	(Chebyshev Polynomials) For $-1\le t\le 1$, define
	\[H_k(t):=\frac{1}{2}(t+i\sqrt{1-t^2})^k+\frac{1}{2}(t-i\sqrt{1-t^2})^k\]
\end{definition}
\begin{lemma}
	For each $k\ge 0$, we have
	\begin{itemize}
		\item[(a)]$H_k$ is a real polynomial of degree $k$.
		\item[(b)]$t^n=\sum_{k\in\Z}\P(S_n=k)H_{\Mod{k}}(t)$
	\end{itemize}
\end{lemma}
\begin{proof}
	To show $(a)$, fix $t\in[-1,1]$ and set $s=\sqrt{1-t^2}$. Observe that
	\[H_k(t)=\frac{1}{2}\sum_{r=0}^k\binom{k}{r}t^{k-r}[(is)^r+(-is)^r]=\frac{1}{2}\sum_{r=0}^{k/2}\binom{k}{2r}t^{k-2r}\psi(s)\]
	where $\psi$ is some real function of $s$.\\
	To show $(b)$ set $z_1=t+is$ and $z_2=t-is$ so that $\Mod{z_1}=\Mod{z_2}=1$ and $z1z2=1$. Then,
	\[H_k(t)=\frac{1}{2}(z_1^k+z_2^k)=H_{-k}(t)\implies \Mod{H_k(t)\le 1}\]
	Now, observe that $t=(z_1+z_2)/2$, so that
	\[t^n=\sum_{k=0}^n\frac{1}{2^n}\binom{n}{k}z_1^kz_2^{n-k}=\sum_{k=0}^n\frac{1}{2^n}\binom{n}{k}z_1^{2k-n}=\frac{1}{2^n}\sum_{r\in\Z}\P(S_n=r)z_1^r\]
	Repeating the same arguments above, we get
	\[t^n=\frac{1}{2^n}\sum_{r\in\Z}\P(S_n=r)z_1^r=\frac{1}{2^n}\sum_{r\in\Z}\P(S_n=r)z_2^r\]
	\[\implies t^n=\frac{1}{2^n}\sum_{r\in\Z}\P(S_n=r)\bigg(\frac{z_1^r+z_2^r}{2}\bigg)=\sum_{r\in\Z}\P(S_n=r)H_{\Mod{r}}(t)\]
\end{proof}

\begin{theorem}
	(Carne-Varopoulos bound) Let $(\Gamma, \mu)$ be a weighted graph. Then, for every $x,y\in V$ and $n\ge 1$
	\[p_n(x,y)\le \frac{2}{\sqrt{\mu_x\mu_y}}\exp\bigg(-\frac{d(x,y)^2}{2n}\bigg)\]
\end{theorem}
\begin{proof}
	Proved in Worksheet 2.
\end{proof}

\subsection*{Large Deviations for Random Walks}
Let $\{\xi_i\}_{i\ge 1}$ be IID $\mathbb{Z}$ valued random variables such that $\E[\xi_1]=\mu$ and $\Var[\xi_1]<\infty$. Define $S_0=0$ and $S_n=\sum_{i=1}^n\xi_i$. Then, the strong law of large numbers (SLLN) and the central limit theorem (CLT) respectively state that
\[\P\bigg(\lim_{n\to\infty}\frac{S_n}{n}=\mu\bigg)=1\text{\ \ and\ \ }\frac{S_n-n\mu}{\sqrt{n}}\overset{d}{\longrightarrow}\SN(0,1)\]
Thus, the CLT loosely states that $S_n \approx n\mu + \sqrt{n}Z$, where $Z\sim\SN(0,1)$.

As an exercise, show that for every $\epsilon > 0$, $\P(A_n^{\epsilon})\to 0$ as $n\to\infty$, where $A_n^{\epsilon}=\{S_n\ge n(\mu + \epsilon)\}$. What is the rate of decay of $\P(A_n^{\epsilon})$ (as $n\to\infty$)?\\
(Hint: $\P(S_n\ge n(\mu + \epsilon))\approx\P(\xi_i > \mu + \epsilon\text{ $\forall$ }1\le i\le n)=[\P(\xi_1 > \mu + \epsilon)]^n\approx e^{-Cn}$ for some $C>0$)

\begin{theorem}
	Let $\{\xi_i\}_{i\ge 0}$ be IID random variables with    $\P(\xi_1=0)=\P(\xi_1=1)=1/2$. Then, for every $a>1/2$,
	\[\lim_{n\to\infty}\frac{1}{n}\log[\P(S_n\ge an)]=-I(a)\]
	where
	\[I(z)=
	\begin{cases}
		\log 2 + a\log a + (1-a)\log a & \text{ if } 0\le z\le 1\\
		\infty & \text{ otherwise}
	\end{cases}
	\]
\end{theorem}

\subsubsection*{Observations:}
\begin{itemize}
	\item[(1)] Minima of $I(z)$ is achieved at $z=1/2$, and the graph increases from $[1/2, 1]$. This implies rate of exponential decay increases as $1/2\to a\to 1$.
	\item[(2)] Symmetry of the function $I(\cdot)$ around $1/2$ suggests that for $a<1/2$, (Requires a proof)
	\[\frac{1}{n}\log[\P(S_n\ge an)]\to -I(a)\]
	\item[(3)] The theorem implies SLLN. The idea of the proof makes use of the following inequality
	\[\P(S_n>(1/2 + \delta)n)\le\exp\{-I(n(1/2+\delta))\}\]
\end{itemize}

\begin{proof} 
	$\\$ 
	If, $a>1$ then, since $S_n$ can be atmost $n$, $\P(S_n > an)=0$ so the result follows. Now, consider $\frac{1}{2} < a \leq 1$, then 
	$$\P(S_n > an) = \sum\limits_{an < k \leq n}\P(S_n= k) = \sum\limits_{an < k\leq n} \binom{n}{k} \frac{1}{2^{n}} =  \frac{1}{2^{n}} \sum\limits_{an < k\leq n} \binom{n}{k} $$
	Let, $Q_n(a) = \max\limits_{an < k \leq n} \dbinom{n}{k}$. So, we have, 
	\begin{equation}
		2^{-n}\,Q_{n}(a) \leq \P(S_n > an) \leq 2^{-n}\,Q_{n}(a)\,(n+1)
	\end{equation}
	First equality follows from the fact that one summand in the $\sum\limits_{an < k\leq n} \binom{n}{k}$ attains maximum and  the second equality follows since, each summand of $\sum\limits_{0 \leq k\leq n} \binom{n}{k}$ is $\leq Q_{n}(a)$. 
	
	\textbf{Claim:} 
	\\
	
	\vspace{-0.25cm}
	For, $\frac{1}{2} < a < 1$, 
	$$\frac{1}{n}\log Q_{n}(a) \xrightarrow[n \rightarrow \infty]{} -a\log a - (1-a)\log (1-a)$$
	Now, from (8.2), 
	\begin{equation}
		-\log 2 + \frac{1}{n}\log Q_{n}(a) \leq \frac{1}{n} \log \P(S_n > an) \leq -\log 2 + \frac{1}{n}\log Q_{n}(a) + \frac{1}{n}\log (n+1)
	\end{equation}
	
	assuming the claim as LHS and RHS of (8.3) goes to $-I(a)$, the result follows. We now prove the claim.
	\\
	
	\vspace{-0.25cm}
	
	\textbf{Proof  of claim:} 
	\\
	
	\vspace{-0.25cm} 
	Since, $a > \frac{1}{2}$, $\max\limits_{an < k \leq n} \dbinom{n}{k} = \dbinom{n}{\ceil{an}}$. Now, from stirling's approximation
	
	$$\binom{n}{\ceil{an}} = \frac{n!}{\lceil an \rceil !(n-\ceil{an})!} \sim \frac{n^ne^{-n}\sqrt{2\pi n}}{\ceil{an}^{\ceil{an}}e^{-\ceil{an}}\sqrt{2\pi \ceil{an}}}\cdot \frac{1}{(n-\ceil{an})^{n-\ceil{an}}e^{n-\ceil{an}}\sqrt{2\pi (n-\ceil{an})}}$$
	
	For, $a > \frac{1}{2}, a<1$; $\ceil{an} \to \infty$ and $n-\ceil{an} \to \infty$ as $n \to \infty$ (Check!) and 
	
	
	\begin{align*}
		\frac{1}{n}\log Q_{n}(a) &\sim \frac{1}{n}\left[ (n+\frac{1}{2})\log n - (\ceil{an}+\frac{1}{2})\log \ceil{an} - (n-\ceil{an} + \frac{1}{2})\log(n-\ceil{an})  - \log (\sqrt{2\pi})\right] \\
		&= \log n + \frac{1}{2n}\log n - \frac{\ceil{an}}{n}\log \ceil{an} - \frac{1}{2n}\log \ceil{an} - \frac{1}{n}\log \sqrt{2\pi}-\frac{n- \ceil{an}}{n} \log (n - \ceil{an}) - \frac{1}{2}\log (n-\ceil{an}) 
	\end{align*}
	the second, fourth, fifth and seventh summand of the above equation tends to 0 as $n$ tends to $\infty$ and from the exercise (?) we have that 
	$$\frac{\ceil{an}}{n}\log \frac{\ceil{an}}{n} \xrightarrow[n \rightarrow \infty]{} a\log a\,\,\,\,\,\,\text{and}\,\,\,\,\,\,\frac{n-\ceil{an}}{n}\log \frac{n-\ceil{an}}{n} \xrightarrow[n \rightarrow \infty]{} (1-a)\log (1-a)$$ 
	which proves the claim.
\end{proof}

\subsection*{Cramer, 1930's}
$\{\xi_i\}_{i\geq 1}$ i.i.d random variables with $\E[\xi_i] = \mu < \infty,\,\E[e^{r\xi_i}] < \infty,\,\forall \, r \in \R$. For any $a > \mu$, 
$$\lim\limits_{n \to \infty} \frac{1}{n}\log \P(S_n > an) = -I(a)$$ where, $I(a) = \sup\limits_{z \in \R}[za-\E[e^{z\xi}]]$

\subsection*{Sanov, 1961 (\texttt{Level 2 of LDP})} 
$\P(S_n > an) = \P \circ S_{n}^{-1}((an,\infty)) \coloneqq \mu_{n}((an,\infty))$
$$-\frac{1}{n}\log \mu_{n}((an,\infty)) \xrightarrow[n \rightarrow \infty]{} \infty$$

\subsection*{Varadhan's LDP setup, 1960's}
Let, $X_n : \Omega \to \R$ be a random variable of $(\Omega, \mathcal{F}, \P)$. $A$ be an event, $\P_{n}(A) \coloneqq \P(S_n \in A)$, then $\P(\cdot)$ is a probability on $\R$. \\

\vspace{-0.25cm}
A sequence $\{\mathcal{P}_n\}_{n \geq 1}$ of probaility measures on $\R$ (can be any metric space $(X,d)$) is said to satisfy large deviation principle with rate n and rate function $I : \R \to \left[0,\infty\right) \cup \{\infty\}$, if 
\begin{enumerate}
	\item $I \not\equiv \infty$, $I$ is lower-semi continuous and has compact level sets.
	\item $\uplim \limits_{n \to \infty}\frac{1}{n} \log \P_{n}(\mathcal{C}) \leq -I(\mathcal{C})\,\forall\,\text{closed sets}\,\, \mathcal{C}$
	\item  $\lowlim \limits_{n \to \infty}\frac{1}{n} \log \P_{n}(\mathcal{O}) \geq -I(\mathcal{O})\,\forall\,\text{open sets}\,\, \mathcal{O}$
\end{enumerate}
where, $A \subseteq \R$, $I(A)= \inf\limits_{y \in A}I(y)$. 

\begin{theorem}
	$\{\mathcal{P}_n\}_{n \geq 1}$ satisfied LDP with rate $n$ then , $I(\cdot)$ is unique.
\end{theorem}
\begin{theorem}[Varadhan's lemma]
	If, $\{\mathcal{P}_n\}_{n \geq 1}$ satisfies LDP with rate $n$ and rate function $I(\cdot)$, let $F_n(x) = \P_{n}(\left(-\infty,n \right])$ for some continuous and bounded above function $F : \R \to \R$, we have 
	$$\int e^{n F(x)} dF_n(x) \xrightarrow[n \to \infty]{} \sup\limits_{x \in \R}[F(x) - I(x)]$$
\end{theorem}
\subsection*{Applications}
For, $\theta \in S^1,\,t \in \R,\,u : S^1 \times \R_{+} \to \R$, 
\begin{align*}
	\frac{\partial u}{\partial t} &= \frac{1}{2}\frac{\partial^{2} u}{\partial \theta^2} + V(\theta)u \\
	u(0,\theta) &= 1
\end{align*}
then,  $$\frac{1}{t}\log u(t,\theta) \xrightarrow[t \to \infty]{} \lambda_1 = \sup\limits_{f \in \cdots} \left\{ \int V(\theta)f(\theta)d\theta - \frac{1}{8}\int \frac{(f'(\theta)^2)}{f(\theta)} d\theta \right\}$$
we can represent this as follows, 
$$u(t,\theta) = \E\,e^{\int\limits_{0}^{t}V(\theta_{s})ds},\,\,\{\theta_{s}\} - \text{brownian motion on}\,S^1$$
\subsection*{Exercises}
\begin{enumerate}
	\item For any $a \in \R$, show that, 
	$$\frac{\ceil{an}}{n} \xrightarrow[n \to \infty]{} a \,\,\,\text{and}\,\,\,\frac{n-\ceil{an}}{n} \xrightarrow[n \to \infty]{} 1-a$$
\end{enumerate}
\end{document}
