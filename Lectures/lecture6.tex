\documentclass[main]{subfiles}

\begin{document}
%Set chapter counter as week-1
\chapPreamble{6}{March 3, 2023}{Harmonic Functions}
%Set chapter name

\lecture{Siva Athreya}{Rahil Miraj, Jainam Khakra}


\textbf{Harmonic Functions:}\\

Let $\Gamma=(V,E, \mu)$ be weighted graph and let $A\subseteq V$, $\bar{A}=A\bigcup\partial A$. Then $f:\bar{A}\rightarrow \mathbb{R}$ is said to be:\\
\textbf{Harmonic} if $\Delta f=0,~i.e.~ (Pf=f)$,\\
\textbf{Super Harmonic} if $\Delta f\leq 0,~i.e.~ (Pf\leq f)$,\\
\textbf{Sub Harmonic} if $\Delta f\geq 0,~i.e.~ (Pf\geq f)$.

\textbf{Examples:}\\
1. For $x,y\in V$ and $A\subseteq V$,
$$g_A(x,y)=\sum_{n=0}^{\infty}p_n^A(x,y)=\frac{\mathbb{E}[L_{\tau_A}^y]}{\mu_y}.$$
$$\Delta g_A(x,y)=-\frac{1}{\mu_x}\mathbf{1_{\{x\}}(y)}=
    \begin{cases}
        -\frac{1}{\mu_x}, & \text{if $y=x$}      \\
        0,                & \text{if $y\neq x$}.
    \end{cases}$$
Therefore, $g_A(x,\cdot)$ is Harmonic in $A\backslash{x}$ and Super Harmonic in $A$.

2. For $z\in V$ and $x\neq z$,$\phi(x)=\mathbb{P}^x(T_z=\infty)$ is Harmonic in $V\backslash{z}$.

\begin{theorem}
    (\textbf{Foster's Criteria/Lyapunov Function}): Let $A\subseteq V$ be a finite set. Then $(\Gamma,\mu)$ is recurrent iff there exists a function $h$, which is:\\
    non negative,
    Super Harmonic on $V\backslash{A}$ and
    $|\{x:h(x)<M\}|<\infty~\forall~ M>0$.
\end{theorem}

\textbf{Proof:}

$(\Rightarrow)$

WLOG, we take $A=\{\rho\}$.
Suppose $\exists h:V\rightarrow [0,\infty)$ such that $h$ is super harmonic on $V\backslash{\{\rho\}}$ and $|\{x:h(x)<M\}|<\infty~\forall ~M>0$.\\
$T_{\rho}=min\{n\geq 0|X_n=\rho\}$, $\{X_n\}_{n\geq 0}$ is a random walk on $(\Gamma,\mu)$.
Let $Y_n=h(X_{n\bigcap T_{\rho}})$ and let $\mathcal{A}_n$ be the observable events upto time $n$. So for $X_0=x$,\\
\begin{align*}
    \mathbb{E}^{x}[Y_n|\mathcal{A}_{n-1}] & =\mathbb{E}^{x}[h(X_{n\bigcap T_{\rho}})|\mathcal{A}_{n-1}]   \\
                                          & =\mathbb{E}^{X_{n-1}}[h(X_{n\bigcap T_{\rho}})],~\text{(SMP)} \\
                                          & =Ph(X_{n-1\bigcap T_{\rho}}),                                 \\
                                          & \leq h(X_{n-1\bigcap T_{\rho}}), ~\text{(super harmonic)},    \\
                                          & = Y_{n-1}.
\end{align*}
\textbf{Super Martingle:} Let $\{Z_n\}_{n\geq 1}$ be a sequence of random variables such that $\mathbb{E}[Z_n]<\infty$. Then $\{Z_n\}_{n\geq 1}$ is a super Martingle if $\mathbb{E}[Z_n|Z_{n-1},Z_{n-2},....Z_1]\leq z_{n-1}$.

\begin{theorem}
    \textbf{(Martingle Convergence Theorem):} Let $Y_n\geq 0$ be Super Martingle, $\exists Y\equiv Y_{\infty}$ such that $Y_n\rightarrow Y_{\infty}$ wp 1 and $\mathbb{E}[Y_{\infty}]\leq\mathbb{E}[Y_0]$ as $n\rightarrow \infty$.
\end{theorem}
So, in our case, $Y_0=h(X_{0\bigcap T_{\rho}})=h(X_0)<\infty$.\\
Therefore, from the above theorem, $Y_{\infty}<\infty$ wp 1.\\
Now, suppose $(\Gamma,\mu)$ is Transient. Then $\exists x\in V\backslash{\{\rho\}}$ such that $\mathbb{P}^x(T_{\rho)}<1$.\\
Let $C_n=\{y\in V\backslash{\{\rho\}}|h(y)\geq n\}~\forall~n\geq 1$. Then $|C_n^c|<\infty$.\\
$N=\{T_\rho=\infty\}\bigcap\{\exists n_k\geq 1: X_{n_k}\in C_k\}$. Let $w\in N$ and $n_k$ be as given by $N$.\\
$$Y_{n_k}=h(X_{n_k \bigcap T_{\rho}}),$$
$$\Rightarrow Y_{n_k}\geq k,$$
$$\Rightarrow N\subseteq \{Y_{\infty}=\infty\},$$
$\Rightarrow\mathbb{P}(Y_{\infty}=\infty)>0$, which contradicts Martingle Convergence Theorem.\\
$\Rightarrow\{X_n\}_{n\geq 0}$ can not be Transient.\\
Hence, $\{X_n\}_{n\geq 0}$ is Recurrent.

$(\Leftarrow)$

Suppose $(\Gamma,\mu)$ is Recurrent. \\
Let $B(\rho,n)=\{x\in V| d(x,\rho)\leq n\}$ and let $h_n:V\rightarrow[0,1]$ such that
$$h_n(x)=\mathbb{P}^x({\tau}_{B(\rho,n)}<T_\rho);~{\tau}_{B(\rho,n)}=T_{B(\rho,n)^c}.$$
$$(SMP)\Rightarrow Ph_n=h_n,~x\neq \rho,$$
$$\Rightarrow \Delta h_n=0,~\forall~x\neq \rho.$$
In particular, $h_n(\cdot)$ is super harmonic in  $V\backslash{\{\rho\}}$.\\
$$\lim_{n\rightarrow\infty}h_n(x)=0,$$
$$h_n(x)=1~\forall~x\in B(\rho,n)^c.$$
$\exists \{n_k\}_{k\geq 1} $ such that $h_{n_k}(x)\leq \frac{1}{2^k}~\forall~x\in B(\rho,n).$\\
Let $x\in V$, $\sum\limits_{k=1}^{\infty}h_{n_k}(x)$ (Ex: $h(\cdot)<\infty$).\\
(I) $h\geq 0$.\\
(II) $x\in V\backslash{\{\rho\}}$,
\begin{align*}
    Ph(x) & =\sum_{x\sim y} \frac{\mu_{xy}}{\mu_x}h(y),                          \\
          & =\sum_{x\sim y} \frac{\mu_{xy}}{\mu_x}\sum_{k=1}^{\infty}h_{n_k}(x), \\
          & =\sum_{k=1}^{\infty}\sum_{x\sim y} \frac{\mu_{xy}}{\mu_x}h_{n_k}(x), \\
          & =\sum_{k=1}^{\infty}Ph_{n_k}(x),                                     \\
          & =\sum_{k=1}^{\infty}h_{n_k}(x)=h(x).
\end{align*}

(III) Let $M>0$ and $U=\{x\in V|h(x)<M\}$.
$$\forall j\geq h_{n_j}(x)=1~\forall~ x \in B(\rho,n_j)^c.$$
\textbf{Claim:} $U\subseteq B(\rho,n_m)^c$.\\
\textbf{Proof:} Given $M>0$, $\exists n_m$ such that\\
$\forall j\geq h_{n_j}(x)=1~\forall~ x \in B(\rho,n_m)^c,$ for $1\leq j\leq M$.\\
$$\sum_{j=1}^{m}h_{n_j}(x)=M~\forall ~x \in B(\rho,n_m)^c.$$
$$\Rightarrow h(x)\geq M ~\forall~ x \in B(\rho,n_m)^c.$$


\noindent \textbf{Theorem (Maximum Principle):}
\\$A \subset V$, connected, $h: V \to \mathbb{R}$ such that $\Delta{h} \geq 0$ on $A$
    \\
    \\a) If $\exists x \in A$ such that $h(x)=\smash{\displaystyle\max_{z \in {A} \cup \partial{A} }} h(z)$ then $h$ is constant on $\overline A$
    \\
    \\b) $|A| < \infty$,  \space $h(z)=\smash{\displaystyle\max_{z \in \partial {A} }}$ \space $h(z)$
    \\
    \\
    \noindent \textbf{Proof}
    \\
    \noindent a) $B=\{y \in \overline A \mid h(y)=h(x)\}$
    \\$B \neq \emptyset$ as $x \in B$
\\$y \in B \cap A$, $z \sim y \Rightarrow z_{0} \in \overline A \Rightarrow z_{0} \in B$
    \\ $h(z_{0}) \leq \smash{\displaystyle\max_{u \in \overline A}}$\space $ h(u)$
    \\
    \\$y \in B \cap A$ and $z \sim y$ then
\\$z_{0} \in \overline A \Rightarrow h(z_{0}) \leq \smash{\displaystyle\max_{u \in \overline A}}$ \space $h(u)=h(x)=h(y)$
    \\
    \\But $\Delta{h(y) \geq 0}$
    \\ $\frac{1}{\mu_{y}} \smash{\displaystyle\Sigma_{z \sim y}}$ \space $\mu_{zy} (h(y)-h(z)) \geq 0$
    \\Hence if $z \sim y \Rightarrow h(z)=h(y)$
    \\Inductively, as $A$ is connected, we have that $B=\overline A$
    \\
    \\b) $|A| < \infty \Rightarrow \exists x_{0} \in \overline A$, $h(x)=\smash{\displaystyle\max_{z \in A}}$ \space $h(z)$
    \\
    \\If $x_{0} \in \partial A \Rightarrow \smash{\displaystyle\max_{u \in \partial A}}$ \space $h(u) = \smash{\displaystyle\max_{z \in \overline A}}$ \space $h(z)$
    \\
    \\If $x_{0} \in A \Rightarrow$ (a) $h$ is constant on $\overline A$ and $\smash{\displaystyle\max_{u \in \partial A}}$ \space $h(u) = \smash {\displaystyle\max_{x \in \overline A}}$ \space $h(x)$ \space
    \\
    \\
    \\
    \noindent \textbf{Liouville Property:}
$(\Gamma, \mu)$ is said to have the Liouville Property if all bounded harmonic functions are constant.
    \\
    \\
    \noindent \textbf{Strong Liouville Property:}
    A graph is said to have the strong Liouville Property if all positive harmonic functions are constant.
    \\
    \\
    \noindent \textbf{Theorem:}
    Let $(\Gamma, \mu)$ be recurrent. Any positive superharmonic function is constant (in particular, $(\Gamma, \mu)$ has the strong Liouville Property).
    \\
    \\
    \noindent \textbf{\underline{Notes for continuation of Martingales}}
    \\

    Let $\{Z_{n}\}$ be random variables on $(\Omega,\mathcal{F},\mathbb{P})$.\\
    Let $E[Z_{n}] < \infty$
    \\
    \\If $E[Z_{n}| Z_{n-1}, Z_{n-2},..., Z_{1}]=Z_{n-1}$, then $\{Z_{n}\}_{n \geq 1}$ is a martingale. \{$h(.)$ is bounded, harmonic, $X_{n}$ r.v on $(\Gamma, \mu)$, $\{h(X_{n})\}_{n \geq 1}$ \}
    \\
    \\If $E[Z_{n}| Z_{n-1}, Z_{n-2},..., Z_{1}] \geq Z_{n-1}$, then $\{Z_{n}\}_{n \geq 1}$ is a submartingale. \{$h(.)$ is bounded, subharmonic, $X_{n}$ r.v on $(\Gamma, \mu)$, $\{h(X_{n})\}_{n \geq 1}$ \}
    \\
    \\If $E[Z_{n}| Z_{n-1}, Z_{n-2},..., Z_{1}] \leq Z_{n-1}$, then $\{Z_{n}\}_{n \geq 1}$ is a supermartingale. \{$h(.)$ is bounded, superbharmonic, $X_{n}$ r.v on $(\Gamma, \mu)$, $\{h(X_{n})\}_{n \geq 1}$ \}
    \\
    \\
    \noindent \textbf{Jensen's Inequality}
    \\
    \\ Let $f:  \mathbb{R} \to \mathbb{R}$ be such that $\forall a \in \mathbb{R}, \exists c \in \mathbb{R}$ such that $f(x) \geq f(a) + c(x-a)$
    \\Then $f$ is said to be a convex function.
    \\
    \\Let $X$ be a random variable on $(\Omega,\mathcal{F},\mathbb{P})$. Then Jensen's Inequality states that $E[f(X)] \geq f(E[X])$ wherever both expectations are well defined.
    \\
    \\
    \\
    \\
    \noindent \textbf{Theorem (Kolmogorov's Maximal Inequality): }Let $\{Z_{n}\}_{n \geq 1}$ be a non-negative submartingale. Then $\mathbb{P} (\smash{\displaystyle\max_{1 \leq i \leq m}}$ \space $Z_{i} \geq a) \leq \frac{E[Z_{m}]}{a}$ \space $\forall a > 0$, $\forall m \geq 1$
    \\
    \\
    \noindent \textbf{Proof:}
    \\Let $m \in \mathbb{N}$ and $a > 0$ be given.
    \\$J=min.\{min.\{n \geq 1 | Z_{n} > a\}, m\}$
\\$\tilde{J} = min.\{n \geq 1 | Z_{n} > a\}$
    \\It is left as an exercise to show that $J$ is a bounded stopping time. $Z_{J} \geq a \Leftrightarrow Z_{n} \geq a$ for some $n \leq m$
    \\
    \\$\mathbb{P}(\smash{\displaystyle\max_{1 \leq i \leq m}}$ \space $Z_{i} \geq a) =\mathbb{P}(Z_{J} \geq a) \leq \frac{E[Z_{J}]}{a}$
\\
\\This follows from the Markov Inequality which can be applied here since $Z_{i} \geq 0$
\\
\\$Z_{J} = \sum_{k=1} ^{m} Z_{k} \mathbf{1}_{J=k} + Z_{m} \mathbf{1}_{\tilde{J}>m}$
    \\
    \\$E[Z_{J}] = \sum_{k=1} ^{m} E[Z_{k} \mathbf{1}_{J=k}] + E[Z_{m} \mathbf{1}_{\tilde{J}>m}]$
\\Since $\{Z_{k}\}_{k \geq 1}$ is a submartingale, we have that the above is less than or equal to
\\$ \sum_{k=1} ^{m} E[E[Z_{m}|\mathcal{A}_{k}]\mathbf{1}_{J=k}] + E[Z_{m} \mathbf{1}_{\tilde{J}>m}]$
    \\The above is equal to
    \\ $\sum_{k=1} ^{m} E[E[Z_{m}\mathbf{1}_{J=k}|\mathcal{A}_{k}]] + E[Z_{m} \mathbf{1}_{\tilde{J}>m}]$ (Since $E[XY|\mathcal{A}_{Y}] = YE[X|\mathcal{A}_{Y}]$)
    \\$=\sum_{k=1} ^{m} E[Z_{m} \mathbf{1}_{J=k}] + E[Z_{m} \mathbf{1}_{\tilde{J}>m}]$
\\$=E[Z_{m} (\sum_{k=1} ^{m} \mathbf{1}_{J=k} + \mathbf{1}_{\tilde{J}>m}]$
    \\$=E[Z_{m}]=0$ \space
\\
\\
\\

\noindent \textbf{Corollary: }Let $\{Z_{m}\}_{m \geq 1}$ be a martingale.
\\
\\1) $E[Z_{m}^2] < \infty$ \space $ \forall m \geq 1, \mathbb{P}(\smash{\displaystyle\max_{1 \leq i \leq m}}$ \space $|Z_{i}| \geq a) \leq \frac{E[|Z_{m}|^2]}{a^2}$
\\
\\2) $Z_{m} \geq 0, \mathbb{P} (\smash{\displaystyle\sup_{n \geq 1}}$ \space $Z_{n} > a) \leq \frac{E[Z_{1}]}{a}$
\\
\\
\noindent \textbf{Proof}
\\For 1) Let $Y_{n}=Z_{n}^2$. Then we apply Kolmogorov's Maximal Inequality. The proof of 2) will be done later. \space
\\
\\
\\
\\
\noindent \textbf{Theorem (Martingale Convergence):}
\\Let $\{Z_{n}\}_{n \geq 1}$ be a martingale and $\smash{\displaystyle\sup_{n \geq 1}}$ \space $E[Z_{n}^2] < \infty$. Then $\exists$ \space $Z$ such that $Z_{n} \rightarrow Z$ w.p. $1$ as $n \rightarrow \infty$
\\
\\
\noindent \textbf{Proof}
\\$f(x)=x^2$ is a convex function. Hence, Jensen's inequality applies to conditional expectation.
    \\$E[Z_{n}^2|Z_{n-1}^2, ... , Z_{1}^2] \geq Z_{n-1}^2$ \space $\forall n \geq 2$
\\
\\From the Tower Property, it follows that
\\$E[Z_{n}^2|Z_{i}^2, ... , Z_{1}^2] \geq Z_{i}^2$ \space for $1 \leq i \leq n$
    \\
    \\Thus, $E[Z_{n}^2] \geq E[Z_{i}^2]$ \space $\forall 1 \leq i \leq n$
    \\In particular, $E[Z_{n}^2] \geq E[Z_{n-1}^2]$ and $\smash{\displaystyle\sup_{n \geq 1}}$ \space $E[Z_{n}^2] < \infty$
    \\
    \\Thus $\exists$ \space $ \alpha > 0$ such that $E[Z_{n}^2] \rightarrow \alpha$ as $ n\rightarrow \infty$
    \\
    \\Let $k \geq 1$, $Y_{m}=Z_{k+m}-Z_{k}$ \space $\forall m \geq 1$
    \\Exercise: $\{Y_{m}\}_{m \geq 1}$ is also a martingale.
    \\
    \\ From the Tower Property, we have that $E[Z_{k+m}Z_{k}]=E[E[Z_{k+m}Z_{k}|\mathcal{A}_{k}]]$ which is equal to $E[Z_{k} E[Z_{k+m}|\mathcal{A}_{k}]]$ which is equal to $E[Z_{k}^2]$ since $\{Z_{n}\}_{n \geq 1}$ is a martingale and $Z_{k}$ is observable in $A_{k}$.
    \\
    \\ Hence $E[Y_{m}^2]=E[Z_{k+m}^2]-E[Z_{k}^2]$
    \\
    \\By Corollary 1,
    \\$\mathbb{P}(\smash{\displaystyle\max_{1 \leq i \leq m}}$ \space $|Y_{i}| > a) \leq \frac{E[Z_{m}^2]}{a^2} = \frac{E[Z_{k+m}^2]-E[Z_{k}^2]}{a^2}$
\\
\\$\mathbb{P}(\smash{\displaystyle\max_{1 \leq i \leq m}}$ \space $|Y_{i}| > a)= \mathbb{P}(\smash{\displaystyle\max_{1 \leq i \leq m}}$ \space $|Z_{i+k}-Z_{k}|>b)$
    \\
    \\Letting $m$ go to $0$ on both sides, we get
    \\$\mathbb{P}(\smash{\displaystyle\cup_{i \geq 1}}$ \space $|Z_{i+k}-Z_{k}| > a) \leq \frac{\alpha - E[Z_{k}^2]}{a^2}$
\\
\\
\\(Exercise:
\\i) $0 \leq \mathbb{P}(\smash{\displaystyle\sup_{i \geq 1}}$ \space $|Z_{i+k}-Z_{k}| > a) \leq \frac{\alpha - E[Z_{k}^2]}{a^2}$
\\
\\ii) $\smash{\displaystyle\lim_{k \rightarrow \infty}}$ \space $\mathbb{P}(\smash{\displaystyle\cup_{i \geq 1}}$ \space $|Z_{i+k}-Z_{k}| > a) = 0$ \space $ \forall a$)
\\
\\
\\
\\From ii) above and the Borel Cantelli Lemma, we have that if
$E=\{\{Z_{k}\}_{k \geq 1}$ is a Cauchy sequence \}
then $\mathbb{P}(E)=1$.
\\$\Rightarrow \exists$ \space $Z$ such that $Z_{n} \rightarrow Z$ w.p. $1$ as $n \rightarrow \infty$. \space
\end{document}
