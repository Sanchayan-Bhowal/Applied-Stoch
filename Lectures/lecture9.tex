\documentclass[main]{subfiles}

\begin{document}
%Set chapter counter as week-1
\chapPreamble{9}{March 31, 2023}{Random walk in trap environment}
%Set chapter name

\lecture{Siva Athreya}{Sanchayan Bhowal, Ramkrishna Samanta}

\section{Continuous time random walk}
In this model the random walker wasits an exponential amount of time to perform a jump like a discrete time random walk. Consider $\{V_i : i\geq 1\}$ to be a collection of independent Exponential$(\lambda)$ random variables. Let $\lambda=1$. Define $T_k$ to be the sum of the first $k$ $V_i$'s. Also, define $N_t$ to be the number of $T_k$ less than $t$. Hence,
\begin{itemize}
    \item $\P(V_i \leq t)=1-e^{-t}$
    \item $T_k=\sum_{i=1}^{k}V_i$
    \item $N_t=\sum_{k=1}^{\infty}\one(T_k\leq t)$
    \item $\{N_t=k\}=\{T_k\leq T_{k+1}\}$
\end{itemize}
\begin{theorem}
    \begin{enumerate}
        \item $N_t\sim$ Poisson$(t)$
        \item $N_t-N_s$ is independent of $N_r$ where $r\leq s \leq t$.
        \item For $0\leq t_0 \leq t_1 \leq \ldots \leq t_n$
              $$
                  \{N_{t_{i+1}}-N_{t_i}: i=0,\ldots, n-1\} \text{ are independent}
              $$
              So, $N_{t_{i+1}}-N_{t_i}\sim$ Poisson$(t_{i+1}-t_i)$
    \end{enumerate}
\end{theorem}
\begin{definition}
    Let $U_n: n\geq 0$ be a random walk on $(\Gamma,\mu)$. Define, a continuous time random walk on $(\Gamma,\mu)$ with rate 1 to be:
    $$Y_t=U_{N_t} \hfill \forall t\geq 0$$.
\end{definition}
\begin{remark}
    The random variable $Y_t$ is a random step function which is right continuous with left limits.
\end{remark}

\section{Random walk in trap environment}
\subsection*{Continuous Time set-up}
Consider the graph $\Z^d$ with natural weights. Let $\{X_t\}_{t\geq 0}$ be a continuous time random walk on $\Z^d$ starting at 0, with rate $\kappa$. Now, let us set up the traps, i.e., for each $y\in \Z^d$ let $N_y\sim$Poisson$(\rho)$. This $N_y$ denote the number of \textit{traps} at $y$. Each trap($Y^{j,y}$) perform a continuous time random walk$(\{Y^{j,y}_t\}_{t\geq 0})$ with rate $\nu$; where $1\leq j \leq N_y$. The random walk gets killed if it meets a trap. There are two ways of killing viz,
\begin{itemize}
    \item [Hard] The walk gets killed upon intersection with any $Y^{j,y}$.
    \item [Soft] At each site $x$ at time $t\geq 0$, define
          $$
              \xi(t,x):=\sum_{y \in \Z^d,1\leq j \leq N_y} \# \{Y^{j,y}\text{ at }x\}.
          $$
          Now $X_t$ gets killed at rate $\gamma\xi(t,x)$ where $\gamma \in \R$.
\end{itemize}
\begin{remark}
    Hard killing infact corresponds to $\gamma=\infty$ case of soft killing.
\end{remark}
The probability of survival is given by
\begin{equation*}
    Z_{\gamma,t}=\E^X[\exp(-\gamma\int_{0}^{t}\xi(s,X(s))ds)]
\end{equation*}
\subsection*{Discrete Time set-up}
Let $\{X_t\}_{t\geq 0}$ be a random walk on $\Z^d$ with natural weights starting at 0. For each $y\in \Z^d$ let $N_y\sim$Poisson$(\rho)$ denotes the number of traps at $y$. Each trap($Y^{j,y}$) perform a lazy random walk$(\{Y^{j,y}_t\}_{t\geq 0})$ on $\Z^d$; where $1\leq j \leq N_y$. The trap kills the random walk with probability $q$ if it meets the random walk; $q \in (0,1)$.
Let $\xi(n,x)$ denote the number of traps at location $x$, i.e.
$$
    \xi(n,x)=\sum_{y \in \Z^d,1\leq j \leq N_y} \delta_x(Y^{j,y}_n).
$$
Assume $X_k$ has survived till $k\leq n$. Given $X_n$ the probability that $X_n$ will survive at time $n$ is $(1-q)^{\xi(n,X_n)}$.
Hence,
\begin{equation}
    \begin{aligned}
        \sigma^X(n,\xi) & =\P(X\text{ has survived till time } n \text{ given } \{Y^{j,y}_m\}_{1\leq j \leq m, y \in \Z^d} \text{ where } m\leq n) \\
                        & =(1-q)^{\sum_{i=1}^{n}\xi(i,X_i)}.
    \end{aligned}
\end{equation}
\section{Pascal's Theorem}
The average survival probability of a given trajectory $X$ is given by $\sigma^X(n)=\E^\xi[(1-q)^{\sum_{i=1}^{n}\xi(i,X_i)}]$.
\begin{theorem}[Pascal]
    The survival probability is maximized by the trajectory $\zero$ where $\zero_k=0$ for every $k \in \N\cup{0}$, i.e,
    $$\sigma^X(n)\leq \sigma^\zero(n).$$
\end{theorem}
\begin{lemma}
    $\sigma^X(n)=\exp (-\lambda \sum_{y \in \Z^d}W_X(n,y))$ where $W_X(n,y)=1-\E^y[1-(1-q)^{\sum_{i=1}^{n}\delta(Y_i^y)}]$. The $Y_i^y$ is a random variable with ditribution same as i.i.d. $Y_i^{j,y}$.
\end{lemma}
\begin{proof}
    Let $X:\N\cup {0} \rightarrow \Z^d$ with $X_0=0$ be the trajectory.
    Now,
    \begin{equation*}
        \begin{aligned}
            \sigma^X(n) & =\E^\xi[(1-q)^{\sum_{i=1}^{n}\xi(i,X_i)}]                                                              \\
                        & =\E^\xi[(1-q)^{\sum_{i=1}^{n}\sum_{y \in \Z^d}\sum_{1\leq j \leq N_y} \delta_{X_i}(Y^{j,y}_n)}]        \\
                        & =\prod_{y \in \Z^d}\E^\xi[\prod_{1\leq j \leq N_y}(1-q)^{\sum_{i=1}^{n}\delta_{X_i}(Y^{j,y}_n)}]       \\
                        & =\prod_{y \in \Z^d}\E^y\E^{N_y}[\prod_{1\leq j \leq N_y}(1-q)^{\sum_{i=1}^{n}\delta_{X_i}(Y^{j,y}_n)}]
        \end{aligned}
    \end{equation*}
\end{proof}
\end{document}