\documentclass[main]{subfiles}

\begin{document}
%Set chapter counter as week-1
\chapPreamble{4}{January 27, 2023}{Discrete Time Martingales}
%Set chapter name

\lecture{Siva Athreya}{Abhiti Mishra, Devesh Bajaj}

Origin is from horse-racing (betting system). The dictionary meaning of the word `martingale' is the harness of a horse. \\
Let $\{Z_n\}_{n \geq 1}$ is a sequence of random variables on $(\Omega, \mathcal{F}, \mathbb{P})$. \\
\begin{definition} A sequence of random variables $\{Z_n\}_{n \geq 1}$ is said to be a Martingale if
  \begin{equation} \label{eq:mar-def}
    \mathbb{E} (Z_{n}|Z_{n-1}=z_{n-1}, \ldots, Z_1=z_1 )= z_{n-1} ~~\forall ~ n \geq 2
  \end{equation} \end{definition}

Things to understand- conditional expectation for discrete and conditional r.v. Reference- Ch6 of Siva's book.\\
Things we will explore-
\begin{enumerate}
  \item Examples of $\{Z_n\}_{n \geq 1}$ that are martingales.\\
  \item How different are martingales from iid sequences and markov chains? \\
  \item How to interpret \ref{eq:mar-def}?
\end{enumerate}

\ex $\{S_n\}_{n \geq 1}$ and $S_0 \equiv 0$.
\begin{equation*}
  X_i=
  \begin{cases}
    1 ,  & w.p ~~ 1/2 \\
    -1 , & w.p ~~1/2  \\
  \end{cases}
\end{equation*}
$$S_n=\sum_{i=1}^n X_i$$
Let $s_{n-1},s_{n-2}, \ldots, s_1 \in \Z$ such that $\mathbb{P} (S_{n-1}=s_{n-1}, \ldots, S_1=s_1)>0$
\begin{align*}
  \mathbb{E} (S_n |S_{n-1}=s_{n-1}, \ldots, S_1=s_1) & = \sum_{k \in \Z} k \mathbb{P} (S_n =k |S_{n-1}=s_{n-1}, \ldots, S_1=s_1)                                                                        \\
                                                     & = \sum_{k \in \Z} k \frac{\mathbb{P}(S_n =k ,S_{n-1}=s_{n-1}, \ldots, S_1=s_1)}{\mathbb{P}(S_{n-1}=s_{n-1}, \ldots, S_1=s_1)}                    \\
                                                     & = \sum_{k \in \Z} k \frac{\mathbb{P}(S_{n-1} +X_n =k ,S_{n-1}=s_{n-1}, \ldots, S_1=s_1)}{\mathbb{P}(S_{n-1}=s_{n-1}, \ldots, S_1=s_1)}           \\
                                                     & = \sum_{k \in \Z} k \frac{\mathbb{P}(X_n =k -s_{n-1} ,S_{n-1}=s_{n-1}, \ldots, S_1=s_1)}{\mathbb{P}(S_{n-1}=s_{n-1}, \ldots, S_1=s_1)}           \\
                                                     & = \sum_{k \in \Z} k \frac{\mathbb{P}(X_n =k-s_{n-1}) \mathbb{P}(S_{n-1}=s_{n-1}, \ldots, S_1=s_1)}{\mathbb{P}(S_{n-1}=s_{n-1}, \ldots, S_1=s_1)} \\
                                                     & = (s_{n-1}+1) \mathbb{P} (X_n=-1)+ (s_{n-1}-1) \mathbb{P} (X_n=1)                                                                                \\
                                                     & = (s_{n-1}+1) \frac{1}{2}+ (s_{n-1}-1) \frac{1}{2} =s_{n-1}                                                                                      \\
\end{align*}
Note that the summations here are ``finite'' sums.\\
As $s_{n-1},\ldots, s_1 \in \Z$ were arbitrary, $\{S_n\}_{n \geq 1}$ is a martingale.\\

\ex $\{X_i\}_{i \geq 1}$ be an iid sequence on $(\Omega, \mathcal{F}, \mathbb{P})$. Let
$Z_n= \prod_{i=1}^n X_i$
and Range($Z_n$) $\subset \R ~~\forall ~~ n \geq 1$. \\
Let $z_{n-1}, \ldots, z_1 \in \R$ such that $\mathbb{P} (Z_{n-1}=z_{n-1}, \ldots, Z_1=z_1)>0$. Then

\begin{align*}
  \mathbb{E} (Z_n |Z_{n-1}=z_{n-1}, \ldots, Z_1=z_1) & = \sum_{k \in Range(Z_n)} k \mathbb{P} (Z_n=k |Z_{n-1}=z_{n-1}, \ldots, Z_1=z_1)                                                                           \\
                                                     & = \sum_{k \in Range(Z_n)} k \frac{\mathbb{P} (Z_n=k ,Z_{n-1}=z_{n-1}, \ldots, Z_1=z_1)}{\mathbb{P} (Z_{n-1}=z_{n-1}, \ldots, Z_1=z_1)}                     \\
                                                     & =  \sum_{k \in Range(Z_n)} k \frac{\mathbb{P} (Z_{n-1}X_n=k ,Z_{n-1}=z_{n-1}, \ldots, Z_1=z_1)}{\mathbb{P} (Z_{n-1}=z_{n-1}, \ldots, Z_1=z_1)}             \\
                                                     & =  \sum_{k \in Range(Z_n)} k \frac{\mathbb{P} (z_{n-1}X_n=k ,Z_{n-1}=z_{n-1}, \ldots, Z_1=z_1)}{\mathbb{P} (Z_{n-1}=z_{n-1}, \ldots, Z_1=z_1)}             \\
                                                     & =  \sum_{k \in Range(Z_n)} k \mathbb{P}(Z_{n-1}X_n=k ) \frac{\mathbb{P} (Z_{n-1}=z_{n-1}, \ldots, Z_1=z_1)}{\mathbb{P} (Z_{n-1}=z_{n-1}, \ldots, Z_1=z_1)} \\
                                                     & = \sum_{u \in S^1, Range(X_n)=S^1} u z_{n-1} \mathbb{P} (X_n=u)                                                                                            \\
                                                     & = z_{n-1} \mathbb{E} [X_n] =z_{n-1}                                                                                                                        \\
\end{align*}
Note that the sums here might be infinite. In the last step we assume $\mathbb{E}[X_i]=1$. Now since $\{z_i\}_{i=1}^{n-1}$ were arbitrary, $\{Z_n\}_{n \geq 1}$ is a martingale. \\

\ex
\begin{equation*}
  X_i=
  \begin{cases}
    2 , & w.p ~~ 1/2 \\
    0 , & w.p ~~1/2  \\
  \end{cases}
\end{equation*}
Then $\mathbb{E} (X_i)=1$. Therefore, $Z_n= \prod_{i=1}^n X_i$ is a martingale. Range ($Z_n$)= $\{2^n,0\}$. Note that the mean stays constant and
$$\mathbb{P}(Z_n=0)=1-\frac{1}{2^n}$$
$$\mathbb{P}(Z_n=2^n)=\frac{1}{2^n}$$
\textbf{Intuition-} The first equation shows that the martingale takes a very low value with very high probability and the second one shows that it takes a very large value with very low probability\\
Idea behind Markov Chains -
$$``X_n | X_{n-1}, \ldots, X_1" \,{\buildrel d \over =}\, X_n|X_{n-1}$$
Idea behind Martingales -
Expected value of $Z_n$ conditioned on the past depends only on $Z_{n-1}$. $\{Z_n\}_{n \geq 1}$ in law could depend on the entire past!

\end{document}
