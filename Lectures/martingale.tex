\documentclass[main]{subfiles}

\begin{document}
%Set chapter counter as week-1
\chapPreamble{4}{January 27, 2023}{Discrete Time Martingales}
%Set chapter name

\lecture{Siva Athreya}{Abhiti Mishra, Devesh Bajaj}

Origin is from horse-racing (betting system). The dictionary meaning of the word `martingale' is the harness of a horse. \\
Let $\{Z_n\}_{n \geq 1}$ is a sequence of random variables on $(\Omega, \mathcal{F}, \mathbb{P})$. \\
\begin{definition} A sequence of random variables $\{Z_n\}_{n \geq 1}$ is said to be a Martingale if
  \begin{equation} \label{eq:mar-def}
    \mathbb{E} (Z_{n}|Z_{n-1}=z_{n-1}, \ldots, Z_1=z_1 )= z_{n-1} ~~\forall ~ n \geq 2
  \end{equation} \end{definition}

Things to understand- conditional expectation for discrete and conditional r.v. Reference- Ch6 of Siva's book.\\
Things we will explore-
\begin{enumerate}
  \item Examples of $\{Z_n\}_{n \geq 1}$ that are martingales.\\
  \item How different are martingales from iid sequences and markov chains? \\
  \item How to interpret \ref{eq:mar-def}?
\end{enumerate}

\ex $\{S_n\}_{n \geq 1}$ and $S_0 \equiv 0$.
\begin{equation*}
  X_i=
  \begin{cases}
    1 ,  & w.p ~~ 1/2 \\
    -1 , & w.p ~~1/2  \\
  \end{cases}
\end{equation*}
$$S_n=\sum_{i=1}^n X_i$$
Let $s_{n-1},s_{n-2}, \ldots, s_1 \in \Z$ such that $\mathbb{P} (S_{n-1}=s_{n-1}, \ldots, S_1=s_1)>0$
\begin{align*}
  \mathbb{E} (S_n |S_{n-1}=s_{n-1}, \ldots, S_1=s_1) & = \sum_{k \in \Z} k \mathbb{P} (S_n =k |S_{n-1}=s_{n-1}, \ldots, S_1=s_1)                                                                        \\
                                                     & = \sum_{k \in \Z} k \frac{\mathbb{P}(S_n =k ,S_{n-1}=s_{n-1}, \ldots, S_1=s_1)}{\mathbb{P}(S_{n-1}=s_{n-1}, \ldots, S_1=s_1)}                    \\
                                                     & = \sum_{k \in \Z} k \frac{\mathbb{P}(S_{n-1} +X_n =k ,S_{n-1}=s_{n-1}, \ldots, S_1=s_1)}{\mathbb{P}(S_{n-1}=s_{n-1}, \ldots, S_1=s_1)}           \\
                                                     & = \sum_{k \in \Z} k \frac{\mathbb{P}(X_n =k -s_{n-1} ,S_{n-1}=s_{n-1}, \ldots, S_1=s_1)}{\mathbb{P}(S_{n-1}=s_{n-1}, \ldots, S_1=s_1)}           \\
                                                     & = \sum_{k \in \Z} k \frac{\mathbb{P}(X_n =k-s_{n-1}) \mathbb{P}(S_{n-1}=s_{n-1}, \ldots, S_1=s_1)}{\mathbb{P}(S_{n-1}=s_{n-1}, \ldots, S_1=s_1)} \\
                                                     & = (s_{n-1}+1) \mathbb{P} (X_n=-1)+ (s_{n-1}-1) \mathbb{P} (X_n=1)                                                                                \\
                                                     & = (s_{n-1}+1) \frac{1}{2}+ (s_{n-1}-1) \frac{1}{2} =s_{n-1}                                                                                      \\
\end{align*}
Note that the summations here are ``finite'' sums.\\
As $s_{n-1},\ldots, s_1 \in \Z$ were arbitrary, $\{S_n\}_{n \geq 1}$ is a martingale.\\

\ex $\{X_i\}_{i \geq 1}$ be an iid sequence on $(\Omega, \mathcal{F}, \mathbb{P})$. Let
$Z_n= \prod_{i=1}^n X_i$
and Range($Z_n$) $\subset \R ~~\forall ~~ n \geq 1$. \\
Let $z_{n-1}, \ldots, z_1 \in \R$ such that $\mathbb{P} (Z_{n-1}=z_{n-1}, \ldots, Z_1=z_1)>0$. Then

\begin{align*}
  \mathbb{E} (Z_n |Z_{n-1}=z_{n-1}, \ldots, Z_1=z_1) & = \sum_{k \in Range(Z_n)} k \mathbb{P} (Z_n=k |Z_{n-1}=z_{n-1}, \ldots, Z_1=z_1)                                                                           \\
                                                     & = \sum_{k \in Range(Z_n)} k \frac{\mathbb{P} (Z_n=k ,Z_{n-1}=z_{n-1}, \ldots, Z_1=z_1)}{\mathbb{P} (Z_{n-1}=z_{n-1}, \ldots, Z_1=z_1)}                     \\
                                                     & =  \sum_{k \in Range(Z_n)} k \frac{\mathbb{P} (Z_{n-1}X_n=k ,Z_{n-1}=z_{n-1}, \ldots, Z_1=z_1)}{\mathbb{P} (Z_{n-1}=z_{n-1}, \ldots, Z_1=z_1)}             \\
                                                     & =  \sum_{k \in Range(Z_n)} k \frac{\mathbb{P} (z_{n-1}X_n=k ,Z_{n-1}=z_{n-1}, \ldots, Z_1=z_1)}{\mathbb{P} (Z_{n-1}=z_{n-1}, \ldots, Z_1=z_1)}             \\
                                                     & =  \sum_{k \in Range(Z_n)} k \mathbb{P}(Z_{n-1}X_n=k ) \frac{\mathbb{P} (Z_{n-1}=z_{n-1}, \ldots, Z_1=z_1)}{\mathbb{P} (Z_{n-1}=z_{n-1}, \ldots, Z_1=z_1)} \\
                                                     & = \sum_{u \in S^1, Range(X_n)=S^1} u z_{n-1} \mathbb{P} (X_n=u)                                                                                            \\
                                                     & = z_{n-1} \mathbb{E} [X_n] =z_{n-1}                                                                                                                        \\
\end{align*}
Note that the sums here might be infinite. In the last step we assume $\mathbb{E}[X_i]=1$. Now since $\{z_i\}_{i=1}^{n-1}$ were arbitrary, $\{Z_n\}_{n \geq 1}$ is a martingale. \\

\ex
\begin{equation*}
  X_i=
  \begin{cases}
    2 , & w.p ~~ 1/2 \\
    0 , & w.p ~~1/2  \\
  \end{cases}
\end{equation*}
Then $\mathbb{E} (X_i)=1$. Therefore, $Z_n= \prod_{i=1}^n X_i$ is a martingale. Range ($Z_n$)= $\{2^n,0\}$. Note that the mean stays constant and
$$\mathbb{P}(Z_n=0)=1-\frac{1}{2^n}$$
$$\mathbb{P}(Z_n=2^n)=\frac{1}{2^n}$$
\textbf{Intuition-} The first equation shows that the martingale takes a very low value with very high probability and the second one shows that it takes a very large value with very low probability\\
Idea behind Markov Chains -
$$``X_n | X_{n-1}, \ldots, X_1" \,{\buildrel d \over =}\, X_n|X_{n-1}$$
Idea behind Martingales -
Expected value of $Z_n$ conditioned on the past depends only on $Z_{n-1}$. $\{Z_n\}_{n \geq 1}$ in law could depend on the entire past!

\chapPreamble{5}{February 3, 2023}{Discrete Time Martingales}
\lecture{Siva Athreya}{Ankan Kar, Atreya Choudhury}

\begin{definition}
    A sequence of random variables $\{Z_n\}_{n \geq 1}$ with $\mathbb{E}[|Z_n|] < \infty$ is said to be martiangle if $\mathbb{E}[Z_n | Z_{n-1}=z_{n-1} , Z_{n-2}=z_{n-2}, . . . , Z_1=z_1] = z_{n-1}$ all are discrete random variables. All $z_i's$ are continuous with appropriate joint deviation.
\end{definition}

\textbf{Statement 4.0.1.\ }Let us define $f : \mathbb{R}^{n-1} \rightarrow \mathbb{R}$ such that $f(z_1, z_2,..., z_{n-1}) = \mathbb{E}[Z_n | Z_{n-1}=z_{n-1} , Z_{n-2}=z_{n-2}, . . . , Z_1=z_1]$.
Then set $Y_n^{(\omega)} = f(z_1^{(\omega)}, z_2^{(\omega)},..., z_{n-1}^{(\omega)})$. We can check that $Y_n$ is a random variable.\\

\emph{\textbf{Properties:}}
\begin{enumerate}
    \item 
    Take $A = Z_{n-1}=z_{n-1},..., Z_1=z-1\}$ for $z_i's \in \mathbb{R}$ where $1 \leq i \leq n$, then;
    \[\omega \in A \Rightarrow Y_n^{(\omega)} = f(z_1,..., z_{n-1})\]
    \item
    Take $L = \{Y_n \leq c\} = \{f(z_1,..., z_{n-1} \leq c\} \ \  \forall c \in \mathbb{R} \Rightarrow L \in \mathcal{A}_{n-1} \equiv$ observable events upto $n-1$.
\end{enumerate}
\[{Statement \  4.0.1.} \iff Y_n \ has \ properties \ 1 \ and \  2\]
Note that $Y_n = \mathbb{E}[Z_n | \mathcal{A}_{n-1}]$. If $Z_n$ is martiangle then $Y_n = Z_{n-1}$.\\

\textbf{Tower Property : \ } $X, Y, Z$ are discrete random variables. $\mathbb{P}(Y=y) > 0$ and $f(y) = \mathbb{E}[X|Y=y]$ then $\mathbb{E}[X|Y] = f(Y)$. $\mathbb{P}(Y=y,Z=z) > 0$, $h(y,z) = \mathbb{E}[X|Y=y,Z=z]$ then $\mathbb{E}[X|Y,Z] = h(Y,Z) \Longrightarrow \mathbb{E}[\mathbb{E}[X|Y,Z]|Y] = \mathbb{E}[X|Y]$.\\

$\mathbb{E}[\mathbb{E}[X|Y,Z]|Y] = \mathbb{E}[h(Y,Z)|Y] := k(Y)$, $\mathbb{E}[X|Y=y] := l(Y)$ \\
Let $Y \in \mathbb{R}$, $\mathbb{P}(Y=y) > 0$, 
\begin{align*}
    k(y)    & = \mathbb{E}[h(Y,Z)|Y=y]\\
            & = \sum\limits_{\underset{t \in Range(Z)}{m \in range(Y)}} h(m,t)\mathbb{P}(Y=m,Z=t|Y=y)\\
            & = \sum\limits_{t \in range(Z)} h(y,t)\mathbb{P}(Z=t,Y=y)\\
            & = \sum\limits_{t \in Range(Z)}(\sum\limits_{c \in range(X)} c\mathbb{P}(X=x|Y=y,Z=t)\mathbb{P}(Z=t|Y=y))\\
            & = \sum\limits_{t \in Range(Z)}\sum\limits_{c \in range(X)} c \frac{\mathbb{P}(X=c,Y=y,Z=t)}{\mathbb{P}(Y=y,Z=t)} \frac{\mathbb{P}(Z=t,Y=y)}{\mathbb{P}(Y=y)}\\
            & = \frac{1}{\mathbb{P}(Y=y)}\sum\limits_{t \in Range(Z)}\sum\limits_{c \in range(X)} c\mathbb{P}(X=c,Z=t,Y=y)\\
            & = \sum\limits_{c \in Range(X)} \frac{c\mathbb{P}(X=c,Y=y)}{\mathbb{P}(Y=y)}\\
            & = \mathbb{E}[X|Y=y] \\
            & = l(y)
\end{align*}
$\Longrightarrow k(Y) = l(Y)$\\
$\Longrightarrow \mathbb{E}[\mathbb{E}[X|Y,Z]|Y] = \mathbb{E}[X|Y]$

\begin{lemma}
Let $\{Z_n\}_{n \geq 1}$ be a martiangle, $1\leq i\leq n$ then; $\mathbb{E}[Z_n | Z_i, Z_{i-1},..., Z_1] = Z_i$
\end{lemma}
\begin{proof}[\textbf{Proof}]
    Fix $i \geq 1 .$ For $n = i+1$,
    \begin{align*}
        \mathbb{E}[Z_{i+1} | Z_i, Z_{i-1},..., Z_1] & = Z_i
    \end{align*}
    Assume for $k \geq 1$, $n=i+k$,
    \begin{align*}
        \mathbb{E}[Z_{i+k} | Z_i, Z_{i-1},..., Z_1] & = Z_i
    \end{align*}
    Then for $n=i+k+1$, by tower property;
    \begin{align*}
        \mathbb{E}[Z_{i+k+1} | Z_i, Z_{i-1},..., Z_1] & = \mathbb{E}[\mathbb{E}[Z_{i+k+1} | Z_{i+k}, Z_{i+k-1},..., Z_1]| Z_i, Z_{i-1},..., Z_1]\\
                                                    & = \mathbb{E}[Z_{i+k} | Z_i, Z_{i-1},..., Z_1]\\
                                                    & = Z_i
    \end{align*}    
\end{proof}


\end{document}
